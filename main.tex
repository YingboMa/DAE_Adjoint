% main.tex
\documentclass[a4paper,9pt,twocolumn]{article}
\usepackage{amsmath,amssymb,amsthm,amsbsy,amsfonts}
\usepackage{physics}
\usepackage{cleveref}
\newcommand{\correspondsto}{\;\widehat{=}\;}
\usepackage{bm}
\usepackage{enumitem} % label enumerate
\newtheorem{theorem}{Theorem}
\theoremstyle{definition}
\newtheorem{definition}{Definition}[section]
\theoremstyle{remark}
\newtheorem*{remark}{Remark}
% change Q.D.E symbol
\renewcommand\qedsymbol{$\hfill \mbox{\raggedright \rule{0.1in}{0.2in}}$}

\begin{document}

\author{Yingbo Ma\\
        \tt{mayingbo5@gmail.com}}
\title{Adjoint of differential-algebraic equations}
\date{January 23, 2020}

\maketitle

\section{Notation}
$M$ denotes a mass matrix. $u$ denotes states. $p$ denotes parameters. $s$
denotes $\pdv{u}{p}$ (sensitivity). $J$ denotes $\pdv{f}{u}$ (Jacobian of the
differential equation with respect to the states).

\section{Adjoint derivation}
Given a function $u$ that satisfies the differential-algebraic equation
\begin{equation} \label{eq:de}
  M\dot{u} = f(u(t), p, t),
\end{equation}
and a cost function (functional)
\begin{equation}
  G[u] = \int_{t_0}^{t_1} g(u(p, t), p) \dd{t}
\end{equation}
in many applications, we may wish to compute the gradient of the cost function
with respect to the parameters $\dv{G}{p}$ to perform optimizations. Na\"ively,
we could apply Leibniz rule for integration an obtain:
\begin{align}
  \dv{G}{p} &= \dv{p} \int_{t_0}^{t_1} g(u(p, t), p) \dd{t} \\
            &= \int_{t_0}^{t_1} \dv{p} g(u(p, t), p) \dd{t} \\
            &= \int_{t_0}^{t_1} \pdv{g}{u}\pdv{u}{p} + \pdv{g}{p} \dd{t} \\
            &= \int_{t_0}^{t_1} \pdv{g}{u}s + \pdv{g}{p} \dd{t}
            \label{eq:dgdp}
\end{align}
It could be very computationally intensive to get the $s$ term, because
we would have to solve another differential equation:
\begin{equation} \label{eq:forward_sens}
  M\dot{s} = Js + \pdv{f}{p}.
\end{equation}

To alleviate the computational cost, we could add a ``$0$'' to $\dv{G}{p}$, and
get:
\begin{align}
  \dv{G}{p} = &\dv{I}{p} \\
  = &\int_{t_0}^{t_1} \pdv{g}{u}s + \pdv{g}{p} \dd{t} \\
  &-\int_{t_0}^{t_1} \lambda^*\qty(\underbrace{M\dot{s} -
  Js - \pdv{f}{p}}_{0 \quad\cref{eq:forward_sens}}) \dd{t}
\end{align}
Now, we have introduced the $\dot{s}$ term to the gradient expression, we can
use the classic technique of integration by parts to simplify the expression
further. The $\lambda^* M\dot{s}$ term can be written as:
\begin{align}
  \int_{t_0}^{t_1} \lambda^* M\dot{s} \dd{t} = \eval{\lambda^* Ms}_{t_0}^{t_1} -
  \int_{t_0}^{t_1} \dot{\lambda}^* Ms \dd{t}.
\end{align}
The gradient expression after grouping is then:
\begin{align} \label{eq:simplified_sens}
  \dv{G}{p} &= \int_{t_0}^{t_1} \pdv{g}{u}s + \pdv{g}{p}
  + \dot{\lambda}^* Ms + \lambda^* Js+ \lambda^* \pdv{f}{p} \dd{t}
              - \eval{\lambda^* s}_{t_0}^{t_1} \nonumber \\
            &= \int_{t_0}^{t_1} \qty(\pdv{g}{u} + \dot{\lambda}^*M  + \lambda^* J)s
              +\lambda^* \pdv{f}{p}  + \pdv{g}{p} \dd{t}
              - \eval{\lambda^* s}_{t_0}^{t_1}
\end{align}
It becomes obvious that we can impose the condition
\begin{equation} \label{eq:cond}
  \pdv{g}{u} + \dot{\lambda}^*M + \lambda^* J = \bm{0} \qq{and}
  \lambda(t_1) = \bm{0},
\end{equation}
to make \cref{eq:simplified_sens} independent of $s$.

After rearranging \cref{eq:simplified_sens} and \cref{eq:cond}, we have
\begin{align}
  M^*\dot{\lambda} &= -J^* \lambda - \qty(\pdv{g}{u})^*, \quad \lambda(t_1) = \bm{0} \\
  \dv{G}{p} &= \int_{t_0}^{t_1} \lambda^* \pdv{f}{p}  + \pdv{g}{p} \dd{t}
\end{align}


\end{document}
