% main.tex
\documentclass[a4paper,9pt]{article}
\usepackage{amsmath,amssymb,amsthm,amsbsy,amsfonts}
\usepackage{systeme}
\usepackage{physics}
\usepackage{cleveref}
\newcommand{\correspondsto}{\;\widehat{=}\;}
\usepackage{bm}
\usepackage{enumitem} % label enumerate
\newtheorem{theorem}{Theorem}
\theoremstyle{definition}
\newtheorem{definition}{Definition}[section]
\theoremstyle{remark}
\newtheorem*{remark}{Remark}
% change Q.D.E symbol
\renewcommand\qedsymbol{$\hfill \mbox{\raggedright \rule{0.1in}{0.2in}}$}

\begin{document}

\author{Yingbo Ma\\
        \tt{mayingbo5@gmail.com}}
\title{Sensitivity of differential-algebraic equations}
\date{March 12, 2020}

\maketitle

\section{Notation}
\begin{itemize}
  \item $M \in \mathbb{C}^{m\times m}$ denotes a mass matrix.

  \item $t_0 < t_1\in \mathbb{C}$ denote constants.

  \item $t \in \mathbb{C}$ denotes independent variable.

  \item $p\in \mathbb{C}^n$ denotes parameters.

  \item $u(p, t): \mathbb{C}^n \times \mathbb{C}\mapsto \mathbb{C}^m$ denotes
    states.

  \item $f\qty(u(p, t), p, t): \mathbb{C}^m \times \mathbb{C}^n \times
    \mathbb{C}\mapsto \mathbb{C}^{m}$ denotes the right-hand-side of a
    differential equation.

  \item $S(p, t): \mathbb{C}^n \times \mathbb{C}\mapsto \mathbb{C}^{m\times n}$
    denotes $\pdv{u}{p}$ (sensitivity with respect to the states).

  \item $J\qty(u(p, t), p, t): \mathbb{C}^m \times \mathbb{C}^n \times
    \mathbb{C} \mapsto \mathbb{C}^{m\times m}$ denotes $\pdv{f}{u}$ (Jacobian of
    the differential equation with respect to the states).

  \item $g(u(p, t), p, t): \mathbb{C}^m \times \mathbb{C}^n \times \mathbb{C}
    \mapsto \mathbb{C}$ is a cost function that is sufficiently smooth.

  \item $\lambda(u(p, t), p, t, t_0, t_1): \mathbb{C}^m \times \mathbb{C}^n
    \times \mathbb{C}\times \mathbb{C}\times \mathbb{C} \mapsto \mathbb{C}^n$
    denotes Lagrange multiplier.

  \item $\lambda_\tau$ denotes $\pdv{\lambda}{\tau}$.
\end{itemize}

\section{Introduction and Forward Sensitivity Analysis}
In many applications, we may wish to compute the gradient of the continuous cost
function
\begin{equation}
  G[u] = \int_{t_0}^{t_1} g(u(p, t), p, t) \dd{t}
\end{equation}
with respect to the parameters $\dv{G}{p}$, where $u$ is a function that
satisfies the differential-algebraic equation
\begin{align} \label{eq:de}
  M\dot{u} &= f(u(p, t), p, t), \\
  u(t_0) &= u_0(p).
\end{align}
Na\"ively, we could apply Leibniz rule for integration and obtain:
\begin{align}
  \dv{G}{p} &= \dv{p} \int_{t_0}^{t_1} g(u(p, t), p) \dd{t} \\
            &= \int_{t_0}^{t_1} \dv{p} g(u(p, t), p) \dd{t} \\
            &= \int_{t_0}^{t_1} \pdv{g}{u}\pdv{u}{p} + \pdv{g}{p} \dd{t} \\
            &= \int_{t_0}^{t_1} \pdv{g}{u}S + \pdv{g}{p} \dd{t}
            \label{eq:dgdp}
\end{align}
We can get $S$ by differentiating \cref{eq:de} with respect to $p$ both sides,
\begin{align}
  &\dv{p}\dv{t}Mu = \dv{p}f(u(p, t), p, t) \\
  \implies &M\dv{t}\dv{u}{p} = \pdv{f}{u}\dv{u}{p} + \pdv{f}{p} \\
  \implies &M\dot{S} = JS + \pdv{f}{p} \label{eq:forward_sens}.
\end{align}
\cref{eq:forward_sens} is often referred as the forward sensitivity equation.
It is apparent that computing $S$ can be intractable when there are large number
of parameters, because we would have to solve a $m\times n$ differential
equation.

\section{Adjoint Sensitivity Analysis with Continuous Cost Function}
To alleviate the computational cost of the forward sensitivity equation, we
could add a ``$0$'' to $\dv{G}{p}$, and get:
\begin{align}
  \dv{G}{p} = &\dv{I}{p} \\
  = &\int_{t_0}^{t_1} \pdv{g}{u}S + \pdv{g}{p} -
  \lambda^*\qty(\underbrace{M\dot{S} -
  JS - \pdv{f}{p}}_{0 \quad\cref{eq:forward_sens}}) \dd{t}.
\end{align}
The motivation of this step is to introduce $\dot{S}$ and an
\textbf{\emph{arbitrary}} $\lambda$ function, and the hope is to use the classic
technique of integration by parts to group the $S$ terms and choose the
$\lambda$ function such that the gradient is \textbf{\emph{independent}} of
$S$.

After integration by parts, the $\lambda^* M\dot{S}$ term is
\begin{align}
  \int_{t_0}^{t_1} \lambda^* M\dot{S} \dd{t} = \eval{\lambda^* MS}_{t_0}^{t_1} -
  \int_{t_0}^{t_1} \dot{\lambda}^* MS \dd{t}.
\end{align}
The gradient expression after grouping is then:
\begin{align} \label{eq:simplified_sens}
  \dv{G}{p} &= \int_{t_0}^{t_1} \pdv{g}{u}S + \pdv{g}{p}
  + \dot{\lambda}^* MS + \lambda^* JS+ \lambda^* \pdv{f}{p} \dd{t}
              - \eval{\lambda^* MS}_{t_0}^{t_1} \nonumber \\
            &= \int_{t_0}^{t_1} \qty(\pdv{g}{u} + \dot{\lambda}^*M  + \lambda^*
            J)S +\lambda^* \pdv{f}{p}  + \pdv{g}{p} \dd{t}
              - \eval{\lambda^* MS}_{t_0}^{t_1}
\end{align}
It becomes obvious that we can impose the condition
\begin{equation} \label{eq:cond}
  \pdv{g}{u} + \dot{\lambda}^*M + \lambda^* J = \bm{0} \qq{and}
  \lambda(t_1)^* M = \bm{0}^*,
\end{equation}
to make \cref{eq:simplified_sens} independent of $S$.

After rearranging \cref{eq:simplified_sens} and \cref{eq:cond}, we have
\begin{align}
    M^*\dot{\lambda} &= -J^* \lambda - \pdv{g}{u}^*, \label{eq:adj_eq_cont}\\
    M^* \lambda(t_1) &= \bm{0} \label{eq:init_cont}\\
    \dv{G}{p} &= \int_{t_0}^{t_1} \lambda^* \pdv{f}{p}  + \pdv{g}{p} \dd{t}
    + \eval{\lambda^* MS}_{t=t_0}. \label{eq:sens_int}
\end{align}
Here, we want to remark that the $\eval{\lambda^* MS}_{t=t_0}$ term is zero if
the initial condition of \cref{eq:de} is independent of the parameters.

\section{Adjoint Sensitivity Analysis with Discrete Cost Function}
In many cases, we only want to compute sensitivity at specific a time point
$\tau$, i.e. $\eval{\pdv{g(u(p, t), p, t)}{p}}_{t=\tau}$. In these cases, we
call the cost function discrete.

To reuse our previous result, we need to find a way to relate $\dv{g}{p}$ to
$\dv{G}{p}$. There are two options, using Dirac delta distribution or using
Leibniz integration rule. We pick the latter because it is more straight
forward.

Note we have
\begin{align}
  \dv{p} \dv{\tau} G[u] &= \dv{p} \dv{\tau} \int_{t_0}^{\tau} g(u(p, t), p, t) \dd{t} \\
                        &= \dv{p}\qty[g(u, p, \tau) + \int_{t_0}^{\tau}
                        \pdv{g}{\tau} \dd{t}] \\
                        &= \dv{g}{p} + \int_{t_0}^{\tau}
                        \pdv{g}{\tau}{p} \dd{t}, \label{eq:dgdp1}
\end{align}
and
\begin{align}
  \dv{\tau} \dv{p} G[u] &= \dv{\tau} \qty[\int_{t_0}^{\tau} \lambda^* \pdv{f}{p} +
  \pdv{g}{p} \dd{t} + \eval{\lambda^* MS}_{t=t_0}] \qq{\cref{eq:sens_int}}\\
                        &= \eval{\qty(\lambda^* \pdv{f}{p} + \pdv{g}{p})}_{t=\tau} +
                        \int_{t_0}^\tau \lambda_\tau^* \pdv{f}{p} +
                        \underbrace{\lambda^* \pdv{f}{p}{\tau}}_{=0}\dd{t}
                        + \int_{t_0}^\tau \pdv{g}{\tau}{p} \dd{t} +
                        \eval{\lambda_\tau^* MS}_{t=t_0}. \label{eq:dgdp2}
\end{align}
By comparing \cref{eq:dgdp1} and \cref{eq:dgdp2}, we can conclude that
\begin{equation} \label{eq:sens_int_dis}
  \dv{g}{p} = \eval{\qty(\lambda^* \pdv{f}{p} + \pdv{g}{p})}_{t=\tau} +
  \int_{t_0}^\tau \lambda_\tau^* \pdv{f}{p} \dd{t} +
  \eval{\lambda_\tau^* MS}_{t=t_0}.
\end{equation}

Now we need $\lambda_\tau$ to compute the sensitivity integral
\cref{eq:sens_int_dis}. It can be obtained by differentiating
\cref{eq:adj_eq_cont},
\begin{align}
  M^*\dot{\lambda_\tau} &= -J^* \lambda_\tau - \pdv{g}{\tau} \\
                               &= -J^* \lambda_\tau \qq{$g$ does not
                               depend on $\tau$}. \label{eq:adj_eq_dis}
\end{align}

We want to remark that from \cref{eq:adj_eq_dis}, we can see that the
initialization of $\lambda_\tau$ cannot be trivial, since
$\lambda_\tau(\tau) = \bm{0}$ can only result in uninteresting dynamics.

It is important to note that $\lambda$ dependents on not only $t$ but also
$\tau$ with a discrete cost function, since $\lambda_\tau$ is non-zero.

To obtain the initialization of $\eval{\lambda_\tau}_{t=\tau}$, we can
differentiate $\eval{\lambda(t, \tau)}_{t=\tau}$ in the constraint
\cref{eq:init_cont} with respect to $\tau$,
\begin{equation}
  \dv{\tau}\qty(\eval{\lambda(t, \tau)^* M}_{t=\tau}) =
  \qty(\underbrace{
    \eval{\pdv{\lambda}{t}\pdv{t}{\tau}}_{t=\tau}}
  _{\dot{\lambda}})^*
  M + \eval{\lambda_\tau^* M}_{t=\tau} = \bm{0}^*.
\end{equation}
Hence, we have
\begin{align}
  \eval{\lambda_\tau^* M}_{t=\tau}
  &= -\eval{\dot{\lambda}^* M}_{t=\tau} \\
  &= \eval{\qty(J^* \lambda + \pdv{g}{u}^*)^*}_{t=\tau} \qq{\cref{eq:adj_eq_cont}}.
\end{align}

Together, the system of equations is,
\begin{align}
    M^*\dot{\lambda_\tau} &= -J^* \lambda_\tau \\
    \eval{M^* \lambda_\tau}_{t=\tau} &= \eval{\qty(J^* \lambda +
    \pdv{g}{u}^*)}_{t=\tau} \\
    \dv{g}{p} &= \eval{\qty(\lambda^* \pdv{f}{p} + \pdv{g}{p})}_{t=\tau} +
                          \int_{t_0}^\tau \lambda_\tau^* \pdv{f}{p} \dd{t} +
                          \eval{\lambda_\tau^* MS}_{t=t_0}.
\end{align}

\subsection{Example: Index-I differential-algebraic equation}
To handle singular mass matrix, we need to split the problem system into
differential variables $\{\cdot\}^d$ and algebraic
variables $\{\cdot\}^a$, so that $\widetilde{M}$ is fully ranked,
\begin{align}
  \mqty(\widetilde{M} & 0 \\ 0 & 0) \mqty(\dot{u}^d\\ \dot{u}^a) &= \mqty(f(u^d,
  u^a, p, t)
  \\ h(u^d, u^a, p, t)) \\
  u^d(t_0) &= u_{d0}(p).
\end{align}
We have
\begin{equation}
  J^* = \mqty(\pdv{f}{u^d}^* & \pdv{h}{u^d}^* \\ \pdv{f}{u^a}^* &
  \pdv{h}{u^a}^*).
\end{equation}
The initialization step is
\begin{equation}
  \mqty(\widetilde{M}^* & 0 \\ 0 & 0) \mqty(\lambda_\tau^d\\ \lambda_\tau^a) =
  \mqty(\pdv{f}{u^d}^* & \pdv{h}{u^d}^* \\ \pdv{f}{u^a}^* &
  \pdv{h}{u^a}^*) \mqty(\bm{0}\\ \lambda^a) + \mqty(\pdv{g^d}{u}^* \\
  \pdv{g^a}{u}^*).
\end{equation}
Expanding the equation we have
\begin{align}
  \widetilde{M}^* \lambda_\tau^d &= \pdv{h}{u^d}^* \lambda^a + \pdv{g^d}{u}^*
  \label{eq:ex1_sys1} \\
  \bm{0} &= \pdv{h}{u^a}^* \lambda^a + \pdv{g^a}{u}^* \implies
  \lambda^a = -\qty(\pdv{h}{u^a}^*)^{-1}\pdv{g^a}{u}^* \label{eq:ex1_sys2}
\end{align}
Plugging \cref{eq:ex1_sys2} into \cref{eq:ex1_sys1}, we obtain the
initialization of $\lambda_\tau$
\begin{equation}
  \widetilde{M}^* \lambda_\tau^d(\tau) = \eval{\qty(-\pdv{h}{u^d}^*
    \qty(\pdv{h}{u^a}^*)^{-1}\pdv{g^a}{u}^* +
  \pdv{g^d}{u}^*)}_{t=\tau}.
\end{equation}

\nocite{cao2003adjoint}

\bibliography{reference.bib}
\bibliographystyle{siam}

\end{document}
